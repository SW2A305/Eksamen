\section{Udvikling af programmet}
%\subsection{Udviklingsprocess}
%\begin{frame}{Udviklingsprocess}
%	\begin{itemize}
%		\item Forskellige metoder
%			\begin{itemize}
%				\item Test Driven Development
%				\item Agile Programering
%				\item Pair Programming
%			\end{itemize}
%		\item Forskellen mellem den reele udvikling og den ønskede.
%	\end{itemize}
%\end{frame}
\subsection{Udviklingsprocess}
\begin{frame}
  \begin{itemize}
    \item Test Driven Development(TDD)
    \item Agil systemudvikling
    \item Ambitioner kontra aktuel arbejdsprocess
  \end{itemize}
\end{frame}

\subsection{Produktkrav}
\begin{frame}{Produktkrav}

\begin{itemize}
	\item Produktkrav er en række funktionelle krav, som danner baggrund for selve programmet.
	\item User stories, fra agil systemudvikling.
\end{itemize}
\end{frame}

\subsubsection{User stories}
\begin{frame}{User stories}
``Som en \textbf{<rolle>} kan jeg \textbf{<mål/ønske>} [så jeg kan \textbf{<fordel>}]'' \newline


% Vores indeholder ingen opfyldelses kriterier, skal dette kommenteres?
\begin{itemize}
\item<2-7>[] \textbf{Udvalgte user stories:}
	\item<2-2,7> ``Som en \textbf{bruger} kan jeg reservere både.''
	\item<3-3,7> ``Som en \textbf{bruger} kan jeg ændre eller slette mine reservationer hvis jeg ombestemmer mig.''
	\item<4-4,7> ``Som en \textbf{elev} kan jeg se informationer omkring sejlerskolen så jeg kan følge mine fremskridt, se fremtidige lektioner osv.''
	\item<5-5,7> ``Som en \textbf{underviser} kan jeg se og ændre brugers fremskridt i sejlerskolen så brugerne bliver opdateret.''
	\item<6-6,7> ``Som \textbf{administrator} vil jeg være i stand til at oprette/fjerne/redigere begivenheder.''
\end{itemize}
\end{frame}


\subsubsection{Prioriterede krav}
\begin{frame}{Prioriterede krav}
\textbf{Overordnedne krav i prioriteret rækkefølge:}
\begin{enumerate}
  \item<+-> Medlemshåndtering
  \item<+-> Brugerlogin
  \item<+-> Undervisningsorganisering 
  \item<+-> Bådreservation
  \item<+-> Logbog
  \item<+-> Begivenhedsadministration og visning.
\end{enumerate}
\end{frame}

\subsection{Programopbygning}
\begin{frame}{Programopbygning}
	\begin{figure}[h]
	\centering
	\tikzstyle{lille} = [rectangle, minimum width=2cm, minimum height=1.0cm,text centered, draw=black, fill=blue!30]
	\tikzstyle{invi} = [draw, rectangle, minimum height=2cm, minimum width=2cm]
	\tikzstyle{line} = [draw]
	\tikzstyle{arrow} = [thick,->,>=stealth]
	\begin{tikzpicture}[node distance = 1.5cm]
	%noderne (objekterne) laves
	\node (uixaml) [lille] {UI XAML};
	\node (cb) [lille, below of=uixaml] {CodeBehind};
	\node (invi1) [invi,draw=none,below of=cb] {};
	\node (model) [lille, below of=invi1] {Model};
	\node (idal) [lille, right of=invi1] {IDal};
	\node (sqldal) [lille, right=0.5cm of idal] {SQLite Dal};
	\node (efdal) [lille, below of=sqldal] {EF Dal};
	\node (sql) [lille, right=0.5cm of sqldal] {SQLite};
	\node (ef) [lille, below of=sql,align=center] {Entity\\ Framework};
	\node (mockdal) [lille, above of=sqldal] {Mock Dal};

	%pointers laves
	\draw [line] (uixaml) -- (cb);
	\draw [line] (cb) -| (idal);
	\draw [dashed] (cb)-- (model);
	\draw [line] (model) -| (idal);
	\draw [line] (idal) -- (sqldal);
	\draw [line] (idal) to [bend right] (efdal);
	\draw [line] (sqldal) -- (sql);
	\draw [line] (efdal) -- (ef);
	\draw [line] (idal) to [bend left] (mockdal);
	\draw [line] (mockdal) to [bend left] (sql);
	\end{tikzpicture}
	
	\label{img:Program_flow}
	\end{figure}
	Skematisk sammenhæng over programmets struktur.
\end{frame}